\documentclass[10pt,a4paper]{scrartcl}

\usepackage[utf8x]{inputenc}
\usepackage{ucs}
\usepackage{amsmath}
\usepackage{amsfonts}
\usepackage{amssymb}
\usepackage{subfig}
\usepackage{graphicx}

\setlength{\parskip}{0.5em}

\author{Simon Wallner\\\small{\texttt{me@simonwallner.at}}}
\title{\emph{Sputnik} Project Report}
\subtitle{HCC Project Seminar}


\begin{document}
\maketitle

\begin{abstract}
Abstract
\end{abstract}

\section{Introduction}
1 page

\section{Literature Review}
2 pages

\section{Results}
7-10 pages

Description of the Implementation according to the learning goals:

\begin{enumerate}
\item Create a system that allows the user to interact with an virtual world through the wiimote. This syststem should be intuitive even for untrained users, and have a very low barrier of entry.

\item Explore the technical capabilities of the wiimote and nunchuck controller and put it to good use.

\item Create a meaningful mapping form the virtual world to the sound generation system.

\item Create a sound generation system that allows nuanced an rich musical expression. 5. Develop an architecture streamlines the three stages input - processing - output.
\end{enumerate}

\section{Evaluation}
3-5 pages

Describe the evaluation according to the research questions. Describe the process and the observed results.


\section{Discussion}
3-5 pages

Discuss the results form the evaluation and answer the research questions. 

\begin{enumerate}
\item How can the arc of light/fishing rod metaphor be used for intuitive interaction. How does lag impact the system?
\item What meaningful mappings can be derived from the interaction with and the visualisation of the virtual scene.
\end{enumerate}

\section{Conclusion}
0.5 pages

wrap up the project, summarise findings, and outline future work




% \bibliographystyle{apalike}
% \bibliography{related-work}

\end{document}