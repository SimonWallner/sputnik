\documentclass[10pt,a4paper]{scrartcl}

\usepackage[utf8x]{inputenc}
\usepackage{ucs}
\usepackage{amsmath}
\usepackage{amsfonts}
\usepackage{amssymb}
\usepackage{subfig}
\usepackage{graphicx}

\setlength{\parskip}{0.5em}


\title{Sputnik}
\subtitle{Project Report HCC Project Seminar 2011}
\author{Simon Wallner\footnote{\texttt{me@simonwallner.at}}}

\begin{document}
\maketitle

\begin{abstract}
This project aims to develop and evaluate a 3D environment with which the user can freely interact through an elastic \emph{arc of light/fishing rod} metaphor, to explore, create and interact with virtual sound objects. These sound objects are placed in the scene and react to the user's input by sending MIDI commands to an external audio program thus creating or manipulating the sound.

\end{abstract}

\section{Reading Guide}
Appendices
Code
Prerequisitions
Video
CD
etc...


\section{Introduction}
\subparagraph{The Challenge}
When I embarked on this endeavour, I had not clear vision of where this project was about the be heading. All I really knew was that I was not satisfied by common interfaces used in computer music creation and manipulation. Computer music is around us for quite some time now, but using the average MIDI controller (with buttons, knobs and faders) still feels like last century technology to me. 

What I was craving for, was a system/instrument that I could use in live settings, or impromptu sessions. Just like any real instrument but it should still open up the creative potential of computer music to me and be very expressive.

Above all these, it should also be comprehensible to the audience what and also how it is happening. \emph{How} not in the sense of a technical understanding of the system, but to the extent that makes it possible for onlookers to forge their own mental model of how the sound is created. 

It should be as easy as looking at a guitar player for instance: Even without knowing anything about the acoustics or physics involved, one can infer from the players motions and the resulting sound, that stroking the strings initiates the sound, and that the other hand on the guitar's fretboard somehow influences the pitch.

Another goal was to make it accessible and rather simple to us. This does not necessarily contradict the need of being an expressive instrument. The piano for example, is very accessible and also rather simple to use. One can probably teach any person to play "Alle meine Entchen\footnote{a very simple german children's song}" in a matter of minutes, but to master it and become a virtuoso performer might easily take half a life time.

It is a common problem of \emph{NIMEs} (\emph{New Interface for Musical Expression}), that usually there is just no one who can play it and no compositions exist. \cite{Dobrian2006} So it should be at least accessible enough to be able to get some reasonable sounds out of it, even for dilettantes like me.

And again, above all these, it should be a generally pleasing thing to look at.



\subparagraph{Sputnik}
To direct this challenges I developed a program called \emph{sputnik}. Sputnik is a 3D environment through which the user can freely move and interact with sound creating/manipulating objects. User input comes from a \emph{wiimote} and \emph{nunchuck} controller the user holds in his/her hands. The scene can be navigated by pushing the analogue stick on the nunchuck controller while the camera is controlled by the wiimote pointer.

Object interaction is performed through an \emph{elastic arc of light/fishing rod} metaphor. The user can \emph{hook} onto objects by pointing at them and then \emph{drag} them around the virtual scene. All objects sport a distinct weight and frictional resistance that can be \emph{felt} through the arc of light.


\subparagraph{Questions to Answer}
Through this project I hope to answer or at least contribute some structured thought to the following questions:

\begin{itemize}
\item How well does the \emph{arc of light} metaphor work?
\item Can this interface be considered a \emph{tangible} interface?
\item How well is it suited to control and manipulate sound?
\item Does it facilitate the creation of mental models for both performers and onlookers?
\end{itemize}







\section{Related Work}
2 pages

\section{Results}
7-10 pages

Description of the Implementation according to the learning goals:

\begin{enumerate}
\item Create a system that allows the user to interact with an virtual world through the wiimote. This system should be intuitive even for untrained users, and have a very low barrier of entry.

\item Explore the technical capabilities of the wiimote and nunchuck controller and put it to good use.

\item Create a meaningful mapping form the virtual world to the sound generation system.

\item Create a sound generation system that allows nuanced and rich musical expression.
\item  Develop an architecture that streamlines the three stages input - processing - output.
\end{enumerate}

\section{Evaluation}
3-5 pages

Describe the evaluation according to the research questions. Describe the process and the observed results.


\section{Discussion}
3-5 pages

Discuss the results form the evaluation and answer the research questions. 

\begin{enumerate}
\item How can the arc of light/fishing rod metaphor be used for intuitive interaction. How does lag impact the system?
\item What meaningful mappings can be derived from the interaction with and the visualisation of the virtual scene.
\end{enumerate}

\section{Conclusion}
0.5 pages

wrap up the project, summarise findings, and outline future work




\bibliographystyle{apalike}
\bibliography{related-work}

\end{document}