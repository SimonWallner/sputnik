\documentclass[10pt,a4paper]{scrartcl}

\usepackage[utf8x]{inputenc}
\usepackage{ucs}
\usepackage{amsmath}
\usepackage{amsfonts}
\usepackage{amssymb}
\usepackage{subfig}
\usepackage{graphicx}

\setlength{\parskip}{0.5em}


\title{Sputnik}
\subtitle{Project Report HCC Project Seminar 2011}
\author{Simon Wallner\footnote{\texttt{me@simonwallner.at}}}

\begin{document}
\maketitle

\begin{abstract}
This project aims to develop and evaluate a 3D environment with which the user can freely interact through an elastic \emph{arc of light/fishing rod} metaphor, to explore, create and interact with virtual \emph{sound objects}. These sound objects are placed in the scene and react to the user's input by sending MIDI commands to an external audio program thus creating or manipulating the sound.

\end{abstract}

\section{Reading Guide}
Appendices
Code
Prerequisitions
Video
CD
etc...


\section{Introduction}
Computer music is around us for some time now and through the use of the computer musicians have sheer endless possibilities of musical expression. With this comes the need to constrain and control it to harness its expressive potential. Over the recent years many standard and non-standard interface have been developed, ranging from the ordinary button-fader-nob MIDI interface to more elaborate interfaces and systems like the \emph{reactable}\cite{Jorda2007}, \emph{mixiTUI}\cite{Pedersen2009} or commercial solutions like the \emph{Novation Launchpad}\footnote{\texttt{http://www.novationmusic.com/products/midi\_controllers/launchpad}} or \emph{Native Instruments Maschine}\footnote{\texttt{http://www.native-instruments.com/\#/en/products/producer/maschine/}}.

With the advent of motion based controllers in consumer entertainment systems, marked by the release of the \emph{Wii}\footnote{\texttt{http://de.wikipedia.org/wiki/Wii}} console in late 2006, motion controller became widely and cheaply available. This makes them the ideal tools to explore the realm of \emph{new interfaces for musical expression}.

A common problem of computer music interfaces is that often the process of sound creation is not readily comprehensible. Seeing a performer on stage behind their laptop twisting knobs and adjusting faders might be ambiguous. It can be hard to relate the artist's action to the resulting sounds. This can hinder experience and might go as far as to the point where the audience suspects that an artist just pressed play as interviews show in \cite{Pedersen2009}.



\section{Paper Outline}
what to find where


\section{Related Work}
2 pages

\section{Results}
`7-10 pages

Description of the Implementation according to the learning goals:

\begin{enumerate}
\item Create a system that allows the user to interact with an virtual world through the Wiimote. This system should be intuitive even for untrained users, and have a very low barrier of entry.

\item Explore the technical capabilities of the Wiimote and Nunchuck controller and put it to good use.

\item Create a meaningful mapping form the virtual world to the sound generation system.

\item Create a sound generation system that allows nuanced and rich musical expression.
\item  Develop an architecture that streamlines the three stages input - processing - output.
\end{enumerate}

\section{Evaluation}
3-5 pages

Describe the evaluation according to the research questions. Describe the process and the observed results.


\section{Discussion}
3-5 pages

Discuss the results form the evaluation and answer the research questions. 

\begin{enumerate}
\item How can the arc of light/fishing rod metaphor be used for intuitive interaction. How does lag impact the system?
\item What meaningful mappings can be derived from the interaction with and the visualisation of the virtual scene.
\end{enumerate}

\section{Conclusion}
0.5 pages

wrap up the project, summarise findings, and outline future work




\bibliographystyle{apalike}
\bibliography{related-work}

\end{document}