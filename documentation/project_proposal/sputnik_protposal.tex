\documentclass[10pt,a4paper]{scrartcl}

\usepackage[utf8x]{inputenc}
\usepackage{ucs}
\usepackage{amsmath}
\usepackage{amsfonts}
\usepackage{amssymb}


\author{Simon Wallner\\\texttt{me@simonwallner.at}}
\title{HCC Project Seminar - Project Proposal}
\subtitle{project working title: \emph{Sputnik}}


\begin{document}
\maketitle

\begin{abstract}
This project proposal for the \emph{HCC Project Seminar} should give an overview about the planed work. The goal of the project is to investigate and provide a prototype implementation for an \emph{electronic music instrument} that uses light as an input device.
\end{abstract}

\section{Introduction}

\section{Related Work}
\subsection{GRL}
Projects under the \emph{GRL} (Graffiti Research Lab) umbrella
\subsubsection{3D light graffiti with matrix like effekt}
\footnote{\texttt{http://www.instructables.com/id/How-to-Enter-the-Ghetto-Matrix-DIY-Bullet-Time/}} A camera array of still image consumer cameras is used to capture a long exposure image of the same object from slightly different angles. The images are then composed to a image sequence. This is a technique made famous by the first \emph{Matrix} movie.

\subsubsection{L.A.S.E.R Tag}
\footnote{\texttt{http://graffitiresearchlab.com/projects/laser-tag/}} Virtual graffiti is created with a projector, a laser pointer and a CV system. The graffiti canvas is projected onto a house wall and calibrated. The laser pointer is then used to \emph{draw} on the house' wall. The path of the laser is tracked by the CV system and \emph{virtual paint} is used to draw on the canvas.

\subsection{PiKA PiKA}
\footnote{\texttt{http://tochka.jp/pikapika/}} Light painting and light painting animations from japan.

\subsection{Light Painting with an iPad}
\footnote{\texttt{http://www.todayandtomorrow.net/2010/09/14/ipad-light-drawing-making-future-magic/}} A digitl 3D scene is preprocessed and then played back on an iPad. During the playback the iPad is moved, thus recreating the original shape of the scene in the final image.

\subsection{Reactable}
\texttt{http://www.reactable.com/}




\end{document}